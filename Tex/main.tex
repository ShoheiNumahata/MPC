\documentclass[12pt,a4j,book]{jlreq}

%\usepackage[dvipdfmx]{graphicx}
\usepackage[ipaex]{luatexja-preset}
\usepackage{amssymb}
\usepackage{amsmath}
\usepackage{amsthm}
\usepackage{float}
\usepackage{bm}
\usepackage{here}
\usepackage{ascmac}
\usepackage{enumerate}

% % 「%」は以降の内容を「改行コードも含めて」無視するコマンド
% \usepackage[%
%  dvipdfmx,% 欧文ではコメントアウトする
%  setpagesize=false,%
%  bookmarks=true,%
%  bookmarksdepth=tocdepth,%
%  bookmarksnumbered=true,%
%  colorlinks=true,%リンクに色をつける
%  linkcolor=blue,%リンクの色を指定
%  pdftitle={},%
%  pdfsubject={},%
%  pdfauthor={},%
%  pdfkeywords={}%
% ]{hyperref}
% % PDFのしおり機能の日本語文字化けを防ぐ((u)pLaTeXのときのみかく)
% \usepackage{pxjahyper}

% %ヘッダー
% \usepackage{fancyhdr}
% \pagestyle{fancy}
%     \lhead{\rightmark} %ヘッダ左
%     \rhead{\leftmark} %ヘッダ右.コンパイルした日付を表示
% % \renewcommand{\headrulewidth}{} %ヘッダの罫線
% % \renewcommand{\footrulewidth}{} %フッタの罫線

\begin{document}
        
    \thispagestyle{empty}
    \begin{center}
    修士論文\\% どちらか一方を消してください.
    \vfill
    〇〇〇〇〇題目〇〇〇〇〇\\
    \vfill
    工藤 遼\\
    \vfill
    主指導教員  藤﨑 英一郎\\
    \vfill
    北陸先端科学技術大学院大学\\
    先端科学技術研究科\\
    (〇〇〇〇)\\ %取得希望学位
    \vfill
    令和〇〇年〇月\\ % 学位授与年月
    \vfill
    \end{center}
    \clearpage
    \centerline{Abstract}
    〇〇〇〇〇〇〇〇〇〇〇〇〇〇〇〇〇〇〇
    \clearpage
    \tableofcontents\thispagestyle{empty}%目次
    \listoffigures\thispagestyle{empty}%図の目次
    \listoftables\thispagestyle{empty}%表の目次
    \clearpage
    
    \chapter{はじめに}
    a
    % 〇〇〇〇〇〇〇〇〇〇〇〇〇〇〇〇〇〇\cite{ref1,ref2}
    \chapter{関連研究}
    \section{セクション名}
    \subsection{サブセクション名}
    〇〇〇〇〇〇〇〇〇〇〇〇〇〇〇〇〇〇
    \chapter{提案手法}
    % 〇〇〇〇〇〇〇〇〇〇〇〇〇〇〇〇〇〇 (図 \ref{fig1})
    
    % \begin{figure}
    % \centerline{\includegraphics[clip,width=90mm]{boltzmann.jpg}}
    % \caption{図のキャプション}\label{fig1}
    % \end{figure}
    \chapter{実験・評価}
    \begin{table}
    \centering
    \begin{tabular}{r|rr}
    & a & b\\ \hline
    1& 0.25 & 0.33\\
    2& 0.75 & 0.66\\
    \end{tabular}
    \caption{表のキャプション}\label{table1}
    \end{table}
    \chapter{おわりに}
    
    % \renewcommand{\bibname}{参考文献}
    % \begin{thebibliography}{99}
    % \bibitem{ref1} 著者名.本のタイトル,出版社 (出版年)
    % \bibitem{ref2} 著者1 and 著者2. 論文タイトル, 学会名, pp.zz--ww (2007)
    % \bibitem{ref3} 著者1, 著者2, and 著者3. 論文タイトル,ジャーナル名,vol.xx, no.yy, pp.zz--ww (2003)
    % \end{thebibliography}
\bibliographystyle{unsrt} % 参考文献出力のスタイル
\bibliography{reference} % 拡張子を外したbibファイル


    
\end{document}
